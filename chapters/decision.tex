\chapter*{答辩决议书}
\markboth{答辩决议书}{}
\addcontentsline{toc}{chapter}{答辩决议书}

{\songti \zihao{5}\setlength{\baselineskip}{16bp}
	XXX同学的博士论文研究基于XXXXX的相关技术,选题具有重要的理论意义和应用价值。

	论文的创新成果包括:

	(1)创新成果1

	(2)创新成果2

	(3)创新成果3

	论文结构清晰、写作规范,表明作者已掌握本学科领域坚实宽广的基础理论和系统深入的专门知识,具备了独立从事科学研究的能力。

	答辩过程中,思路清晰,回答问题正确。

	经答辩委员会无记名投票,一致通过XXXX同学博士学位论文答辩,并建议授予其工学博士学位。

	\vspace*{3em}
	\hfill 答辩委员会主席:\hspace*{7em}

	\vspace*{1em}
	\hfill \hspace*{4em}年\hspace*{2em}月\hspace*{2em}日
}