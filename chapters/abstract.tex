\begin{abstract}

	[鼠标左键单击选择该段落,输入替换之。内容为小四号宋体。]中文摘要应将学位论文的内容要点简短明了地表达出来,硕士学位论文一般为500-1000字,博士学位论文一般为1000至2000字。留学生英文版学位论文不少于3000字中文摘要,留学生英文版博士学位论文不少于5000字中文摘要。字体为宋体小四号。内容应包括工作目的、研究方法、成果和结论。要突出本论文的创新点,语言力求精炼。为了便于文献检索,应在本页下方另起一行注明论文的关键词(3-8个),如有可能,尽量采用《汉语主题词表》等词表提供的规范词。图X幅,表X个,参考文献X篇。

	摘要是论文内容的高度概括,应具有独立性和自含性,即不阅读论文的全文,就能通过摘要获得必要的信息。摘要应包括研究目的、内容、方法、结果和结论等,重点是结果和结论。
	摘要的内容要完整、客观、准确,应做到不遗漏、不拔高、不添加。摘要应按层次逐段简要写出,摘要在叙述研究内容、研究方法和主要结论时,除作者的价值和经验判断可以使用第一人称外,一般使用第三人称,采用“分析了……原因”、“研究了……”、“对……进行了探讨”“给出了……结论”等记述方法进行描述。避免主观性的评价意见,避免对背景、目的、意义、概念和一般性(常识性)理论叙述过多。
	摘要需采用规范的名词术语(包括地名、机构名和人名)。对个别新术语或无中文译文的术语,可用外文或在中文译文后加括号注明外文。摘要中应尽量避免使用图、表、化学结构式、非公知公用的符号与术语,不标注引用文献编号。
	博士学位论文摘要应包括以下几个方面的内容:
	(1)论文的研究背景及目的。简洁准确地交代论文的研究背景与意义、相关领域的研究现状、论文所针对的关键科学问题,使读者把握论文选题的必要性和重要性。此部分介绍不宜写得过多,一般不多于400字。
	(2)论文的主要研究方法与研究内容。介绍论文所要解决核心问题开展的主要研究工作以及研究方法或研究手段,使读者可以了解论文的研究思路、研究方案、研究方法或手段的合理性与先进性。
	(3)论文的主要创新成果。简要阐述论文的新思想、新观点、新技术、新方法、新结论等主要信息,使读者可以了解论文的创新性。创新成果注意凝练和综合,一般以2~4项为宜。
	(4)论文成果的理论和实际意义。客观、简要地介绍论文成果的理论和实际意义,使读者可以快速获得论文的学术价值。

	\vspace{50pt}

	关键词是供检索用的主题词条,用显著的字符另起一行,排在摘要的下方。关键词应集中体现论文特色,具有语义性,在论文中有明确的出处,并应尽量采用《汉语主题词表》或各专业主题词表提供的规范词。每篇论文应选取3~8个关键词。

	\noindent\keywords{[请输入关键词(3-8),以分号分隔。]}

\end{abstract}
